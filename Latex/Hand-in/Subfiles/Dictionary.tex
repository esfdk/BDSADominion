\section{Dictionary}
\subsection{General terms}
This section describes the general "out-of-game" terms.
\begin{description}
\item[Achievements] An achievement is token rewarded for funny and/or hard accomplishments within the game.
\item[Card-set] A card-set is 10 different Kingdom cards. Card-sets are used to create a different play experience every time you play.
\item[Dominion] The card-game we are making a virtual representation of. A link to the full rules can be found at Rio Grande Games \cite{dominionRules}.
\item[Draft] Drafting is done by player 1 selecting one Kingdom card to be used in the game, then player 2 selects a Kingdom card, player 3 selects a Kingdom card, player 4 selects a Kingdom card, then back to player 1. This cycle repeats until a set number of Kingdom cards have been selected.
\item[Extensions] Expansion packs add additional types of cards to the pool of cards.
\item[Game Mode] There are different possible game modes: draft, random card selection and predefined card-sets. These are selected before the game starts.
\item[Hot-Seat] Hot-Seat is the act of having 2 or more players play on the same computer. The active player "sits" in the hot-seat while playing, then passing the spot to the next player when his turn ends.
\item[Message Type] Messages of different types can be passed around in our server-client network.
\item[Model-View-Control] Often abbreviated MVC, Model-View-Control is often used to seperate something "showing" data and the actual representation of the data on the disk. Control is usually the middle-link that takes care of the communication between the two.
\item[Picture-based GUI] A pictured-based GUI is a visual representation of the state of the game. The different cards are shown as pictures in the GUI.
\item[Server-Client] In a client-server design, the clients communicate with the server and the server then relays the information it was given by the client to the other clients.
\item[Statistics] Statistics such as number of games played, numbers of games won/lost, and other similar data about gameplay.
\item[Tooltip] A box with text describing something in the GUI in detail.
\item[User] A user is an entity storing statistics and achievements over the course of different games.
\end{description}
\subsection{In-Game terms}
This section describes the types of cards, supply and other "in-game" terms.
\begin{description}
\item[Available] Available Cards are the Cards that can be bought from the Supply.
\item[Action Phase] In an action phase, a player have one Action, which he or she may use to play an Action Card. Playing an action card this way always costs one Action. Cards played may allow a player to receive additional actions. The Action Phase ends when a player has no more Actions left or chooses not to use his or her remaining Actions. 
\item[Buy Phase] When a player's Action Phase ends, the Buy Phase begins. In this, the player receives a "Coin" amount, which is the combined value of all Treasure Cards in his or her hand and any Action Cards, that add "Coins". The player can then use a Buy to buy any Card they want from the Supply. Played Action Cards can allow more Buys. Bought Cards are added to the Discard Stack. After the Buy Phase, the Clean-Up Phase begins
\item[Card] A Card is the basic playing unit in Dominion. Everything you 'own' is represented by a Card in your deck. Cards are primarily added to the deck through the Buy phase. Each Card has a value, which represents what it costs to get during the Buy Phase.
\begin{description}
\item[Curse Card] A Curse Card is a special type of Victory Card, which gives a negative amount of Victory Points. While these cards can technically be bought by any player and added to his or her deck, they are usually given to other players by using Attack Cards against them.
\item[Kingdom Card] Kingdom Cards are what make each game of Dominion unique. With one exception all Cards here are Action Cards (one is a special Victory Card) and there are no Action Cards which are not Kingdom Cards. Each game requires selecting 10 of the 25 Kingdom Cards to use.
\begin{description}
\item[Action Card] An Action Card is used during the Action Phase.
\begin{description}
\item[Attack Card] An Attack Card is a type of Card which affects other players negatively. All Attack Cards are Action Cards and the "Attack" actives when the Card is used as an Action.]
\item[Action-Reaction Card] A Reaction card is used to respond to an Attack by another player. When an Attack Card is used against a player, that player may reveal a Reaction Card from his or her hand and do what the Reaction allows. Only one Reaction Card is in this game, 'Moat', which allows the player to negate the attack used against them.
\end{description}
\item[Kingdom Victory Card] A Kingdom Victory Card is a card that does generally not behave like usual Victory Card, but instead have special effects granting the player Victory Points.
\end{description}
\item[Treasure Card] A Treasure Card adds a number of "Coins" to spend in the Buy Phase. Note that a Treasure Cards value (the price to buy it) are usually different from what they cost to buy.
\item[Victory Card] A Victory Card gives a number of Victory Points at the end of the game. The player with the most Victory Points win the game.
\end{description}
\item[Clean-up Phase] The Clean-up Phase consists of putting all bought Cards, played Cards and Cards remaining in the Hand into the Discard Stack.
\item[Deck] A players Deck is his or her representation in the game. It consists of all the Cards that player started with and have bought during the game. A player's Draw Stack, Discard Stack and Hand is that player's Deck.
\item[Discard Stack] This contains previously played cards and any newly bought cards.
\item[Draw Stack] This contains face-down Cards for a player to draw. When there are no more cards available for a player to draw, the Discard Stack is shuffled and used as a new Draw Stack. Each player have their own Draw Stack and Discard Stack.
\item[Hand] The Hand represents a players current options in the following turn. These are drawn at the start of the game and each player draws a new hand after a turn has finished. When drawing a new hand, it always consists of 5 Cards.
\item[Supply] The Supply consists of 10 types of Kingdom Cards, 3 types of Treasure Cards, 3 types of Victory Cards and Curse Cards.
\item[Round] A game of Dominion consists of a number of rounds. Each Round is divided in to Turns, one for each player.
\item[Trash Stack] Sometimes a Card calls for itself or some other card to be Trashed. This means that it should be completely removed from the game and the Trashed Card is put on to the Trash Stack. All players share the Trash Stack.
\item[Turn] The player usually take turns in clockwise order. A players next Turn will be in the following Round.
\end{description}
