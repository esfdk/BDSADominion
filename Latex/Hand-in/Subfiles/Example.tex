\section{Example}
\label{sec:Examp}
\paragraph{Frederik Roden Lysgaard}
This will be a example of our project, which is a graphical representation of the cardgame "Dominion", published by Rio Grande Games.
The walkthrough will be built up around certain screenshots and will cover the following points:

\begin{itemize}
\item Starting the game, hotseat or LAN.
\item Getting started, what is Dominion really about?
\item The user interface.
\item End of game.
\end{itemize}

\subsection{Starting the game,hotseat or LAN.}
\subsubsection{LAN}
To start the application you need to locate the executable folder which contains the BDSADominion.application launching this will start the application, it's worth noting that the it will take some time the first time your load it since it needs to find some assets. When starting the application you will be presented with a console window asking you to take the role of either Client or Server

\begin{figure}[h!]
\centering
\includegraphics[width=1\linewidth]{subfiles/images/Clientinput}
\caption{Client input}
\end{figure}

One of the players will choose server and will then be able to give his fellow players who chose client, an IP to connect to.
\begin{figure}[h!]
\centering
\includegraphics[width = 1\linewidth]{subfiles/images/startserver}
\caption{start server}
\end{figure}
When the appropriate amount of players has joined the server ( usually 3-4) then the person with the server program runing will
call startgame (command is \textless STGM\textgreater ) and the game will then start.
\begin{figure}[h!]
\centering
\includegraphics[width=1\linewidth]{subfiles/images/startgame}
\caption{start game}
\end{figure}
\begin{figure}[h!]
\centering
\includegraphics[width=1\linewidth]{subfiles/images/startscreen}
\caption{The start screen}
\end{figure}
\subsubsection{Hot-Seat}
This is done almost identically to the procedure for a LAN game the only exception is, that instead of letting other computers create a client, you just run multi instance of the application on your computer like so:
\begin{figure}[h!]
\centering
\includegraphics[width =1\linewidth]{subfiles/images/serverclient}
\caption{hotseat, server-client}
\end{figure}
and after that it's just that same as with LAN games.


\subsection{Getting started, what is Dominion really about?}
Dominion is a deckbuilding game which means, that the object of the game is to build yourself a deck which will give you the best hands, and thereby giving you the edge in getting the most victory points which in the end determines who wins.
I already introduced some of the game specific words and I will now show where they are placed on the playing board and what their responsibility is:

\subsection{The user interface}

\begin{figure}[h!]
\centering
\includegraphics[width =1\linewidth]{subfiles/images/actioncard}
\caption{The playing board}
\end{figure}
\begin{description}
\item[hand]The hand is where you see what you have drawn each turn.In Dominion there's three kind of cards: Treasure, Victory or Action all three kinds can be drawn into this field. If you click on a Action card while it's placed in hand and you got actions left then the card will be moved from the hand to the actionzone.
\item[actionzone] The actionzone is where the actioncards that is played from the hand is shown. Only actioncards can be drawn in this zone. When a turn ends the actionzone will be cleared and the actioncards will all be moved to the discard zone.
\item[discard] The discard zone is where the cards go when they are not in use anymore, you can't click on cards while they are in the discardzone, while in the discardzone the cards can only wait to be shuffled into the deck again.
\item[deck] The deck is where the cards is held until they are drawn.
\item [supply] The supplyzone that are drawn to right side of the GUI is crucial to the game. This is where you can buy your kingdom cards and thereby increase your decks size and strength, as we can see, we draw both the seven static victory/treasure cards and 10 extra kingdomcards, which are also one of our mandatory requirements.
\end{description}

\subsection{End of game}
As stated by the Dominion game rules the game ends when either the Province card pile is empty or three kingdomcard piles are empty. When this happens we check which player has won. If you are indeed the winner then you will be greeted with a lovely congratulations message printed across the screen.
\begin{figure}[h!]
\centering
\includegraphics[width =1\linewidth]{subfiles/images/youarewinner}
\caption{You are the winner!!}
\end{figure}
but if you lose you will be met with disgust.
\begin{figure}[h!]
\centering
\includegraphics[width=1\linewidth]{subfiles/images/youloose}
\caption{You loose!!!}
\end{figure}
So to summarize: Our digital representation of Dominion lets you play with 10 preset kingdom cards in a graphical interface with your friends either over hotseat or local area network.
 