\documentclass[12pt,a4paper,notitlepage]{article}
\usepackage[latin1]{inputenc}
\usepackage[english]{babel}
\usepackage{graphicx}
\usepackage{fullpage}
\usepackage[final]{pdfpages}
\usepackage[font=sl]{caption}
\usepackage{listings}
\usepackage{hyperref}
\hypersetup{
    colorlinks,%
    citecolor=black,%
    filecolor=black,%
    linkcolor=black,%
    urlcolor=black
}

\setlength{\parindent}{0pt}
\setlength{\parskip}{1.8ex plus 0.5ex minus 0.2ex}

\title{\huge{Dominion}\\
\large{Analysis, Design and Software Architecture}
}
\author{
Jakob Melnyk, jmel@itu.dk\\
Christian Jensen, chrj@itu.dk\\
Frederik Lysgaard, frly@itu.dk
}
\date{November 25th, 2011}
\begin{document}
\maketitle
\vfill
\section*{Abstract}
This project is about a virtual representation of the card game Dominion in C\#. Dominion is a turn-based, deck-building game, where the objective is to gather more points than the other players. The game is played by 2 - 4 players.
\clearpage
\tableofcontents
\pagebreak
\section{Requirements}
\subsection{Mandatory}
Must be able to play a full game of Dominion
\begin{itemize}
\item Must support 2 players in Hot-Seat configuration
\item At least 10 Kingdom cards must work
\item The game must be playable in a Picture-based GUI
\end{itemize}
\subsection{Secondary}
High priority
\begin{itemize}
\item Be able to play the game with 3 or more players
\item Be able to use at least 20 Kingdom cards
\item Be able to select Game Mode
\begin{itemize}
\item Be able to play 'First Game' Card-set
\item Be able to play with 10 randomly select Kingdom cards
\end{itemize}
\item Be able to see all Available Kingdom cards without scrolling
\end{itemize}
Medium priority
\begin{itemize}
\item Be able to view a Tooltip when mousing over any Card in the game
\item Be able to play the game over LAN
\item Be able to use all Kingdom cards (from the original version of the game)
\item Be able to play all the Card-sets defined in the original rules
\end{itemize}
Low priority
\begin{itemize}
\item Be able to Draft Kingdom cards
\item Be able to play the game over the Internet
\item Be able to select different screensizes
\item Be able to play in fullscreen
\item Be able to create a User, that is saved across multiple games, with the following information:
\begin{itemize}
\item Statistics
\item Options (if any)
\item Achievements (if implemented)
\end{itemize}
\item Be able to support Extensions of the basic game
\item Implement Achievements for funny and/or hard accomplishments
\end{itemize}
\section{Overview}
This project is about our virtual representation of the card game Dominion. Dominion is a turn-based, deck-building game. The objective of the game is to use Action cards to improve your chances or damage the opponent players and using Treasure cards to buy more powerful Action/Treasure/Victory cards to gain the upper hand. 

We are planning on using a Model-View-Controller architecture. We want to separate our GUI from both the game rules and the state of the game via the controller. In essence we are likely to have a somewhat static model of the rules and a more dynamic and changing model of the state of the current game.

Frederik Lysgaard is the guy responsible for the design of our graphical interface. He is also the best Dominion player in our group. Because of this, he knows a lot of the usual strategies and is our general "go-to" guy when it comes to the tactics of the game.

Christian Jensen is responsible for implementing the way the different cards interact with the state of the game when used. Christian is also the guy who will be looking into the networking portion of the project if/when it becomes relevant.

Jakob Melnyk is responsible for modeling the state of the game and the communication between the GUI and the model (in our model-view-controller architecture). Jakob  Melnyk is also the "version-control-guy", the person with the final word in discussions and the general log-keeper for the group.
\section{Dictionary}
\subsection{General terms}
This section describes the general "out-of-game" terms.
\begin{description}
\item[Achievements] An achievement is token rewarded for funny and/or hard accomplishments within the game.
\item[Card-set] A card-set is 10 different Kingdom cards. Card-sets are used to create a different play experience every time you play.
\item[Dominion] The card-game we are making a virtual representation of. A link to the full rules can be found at Rio Grande Games \cite{dominionRules}.
\item[Draft] Drafting is done by player 1 selecting one Kingdom card to be used in the game, then player 2 selects a Kingdom card, player 3 selects a Kingdom card, player 4 selects a Kingdom card, then back to player 1. This cycle repeats until a set number of Kingdom cards have been selected.
\item[Extensions] Expansion packs add additional types of cards to the pool of cards.
\item[Game Mode] There are different possible game modes: draft, random card selection and predefined card-sets. These are selected before the game starts.
\item[Hot-Seat] Hot-Seat is the act of having 2 or more players play on the same computer. The active player "sits" in the hot-seat while playing, then passing the spot to the next player when his turn ends.
\item[Picture-based GUI] A pictured-based GUI is a visual representation of the state of the game. The different cards are shown as pictures in the GUI.
\item[Statistics] Statistics such as number of games played, numbers of games won/lost, and other similar data about gameplay.
\item[Tooltip] A box with text describing something in the GUI in detail.
\item[User] A user is an entity storing statistics and achievements over the course of different games.
\end{description}
\subsection{In-Game terms}
This section describes the types of cards, supply and other "in-game" terms.
\begin{description}
\item[Available] Available Cards are the Cards that can be bought from the Supply.
\item[Action Phase] In an action phase, a player have one Action, which he or she may use to play an Action Card. Playing an action card this way always costs one Action. Cards played may allow a player to receive additional actions. The Action Phase ends when a player has no more Actions left or chooses not to use his or her remaining Actions. 
\item[Buy Phase] When a player's Action Phase ends, the Buy Phase begins. In this, the player receives a "Coin" amount, which is the combined value of all Treasure Cards in his or her hand and any Action Cards, that add "Coins". The player can then use a Buy to buy any Card they want from the Supply. Played Action Cards can allow more Buys. Bought Cards are added to the Discard Stack. After the Buy Phase, the Clean-Up Phase begins
\item[Card] A Card is the basic playing unit in Dominion. Everything you 'own' is represented by a Card in your deck. Cards are primarily added to the deck through the Buy phase. Each Card has a value, which represents what it costs to get during the Buy Phase.
\begin{description}
\item[Curse Card] A Curse Card is a special type of Victory Card, which gives a negative amount of Victory Points. While these cards can technically be bought by any player and added to his or her deck, they are usually given to other players by using Attack Cards against them.
\item[Kingdom Card] Kingdom Cards are what make each game of Dominion unique. With one exception all Cards here are Action Cards (one is a special Victory Card) and there are no Action Cards which are not Kingdom Cards. Each game requires selecting 10 of the 25 Kingdom Cards to use.
\begin{description}
\item[Action Card] An Action Card is used during the Action Phase.
\begin{description}
\item[Attack Card] An Attack Card is a type of Card which affects other players negatively. All Attack Cards are Action Cards and the "Attack" actives when the Card is used as an Action.]
\item[Action-Reaction Card] A Reaction card is used to respond to an Attack by another player. When an Attack Card is used against a player, that player may reveal a Reaction Card from his or her hand and do what the Reaction allows. Only one Reaction Card is in this game, 'Moat', which allows the player to negate the attack used against them.
\end{description}
\item[Kingdom Victory Card] A Kingdom Victory Card is a card that does generally not behave like usual Victory Card, but instead have special effects granting the player Victory Points.
\end{description}
\item[Treasure Card] A Treasure Card adds a number of "Coins" to spend in the Buy Phase. Note that a Treasure Cards value (the price to buy it) are usually different from what they cost to buy.
\item[Victory Card] A Victory Card gives a number of Victory Points at the end of the game. The player with the most Victory Points win the game.
\end{description}
\item[Clean-up Phase] The Clean-up Phase consists of putting all bought Cards, played Cards and Cards remaining in the Hand into the Discard Stack.
\item[Deck] A players Deck is his or her representation in the game. It consists of all the Cards that player started with and have bought during the game. A player's Draw Stack, Discard Stack and Hand is that player's Deck.
\item[Discard Stack] This contains previously played cards and any newly bought cards.
\item[Draw Stack] This contains face-down Cards for a player to draw. When there are no more cards available for a player to draw, the Discard Stack is shuffled and used as a new Draw Stack. Each player have their own Draw Stack and Discard Stack.
\item[Hand] The Hand represents a players current options in the following turn. These are drawn at the start of the game and each player draws a new hand after a turn has finished. When drawing a new hand, it always consists of 5 Cards.
\item[Supply] The Supply consists of 10 types of Kingdom Cards, 3 types of Treasure Cards, 3 types of Victory Cards and Curse Cards.
\item[Round] A game of Dominion consists of a number of rounds. Each Round is divided in to Turns, one for each player.
\item[Trash Stack] Sometimes a Card calls for itself or some other card to be Trashed. This means that it should be completely removed from the game and the Trashed Card is put on to the Trash Stack. All players share the Trash Stack.
\item[Turn] The player usually take turns in clockwise order. A players next Turn will be in the following Round.
\end{description}
%\section{Example}
%\section{Revision History}
\begin{thebibliography}{9}
\bibitem{dominionRio} http://www.riograndegames.com/games.html?id=278
\bibitem{dominionRules} http://www.riograndegames.com/uploads/Game/Game\_278\_gameRules.pdf
\end{thebibliography}
\end{document}